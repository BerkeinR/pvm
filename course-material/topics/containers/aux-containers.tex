\usetikzlibrary{shadows,shapes.multipart}

\title{Containers}



\begin{document}

\begin{frame}
  \titlepage
\end{frame}

\section{\texttt{std::vector}}

\frame{\tableofcontents[currentsection]}

\begin{frame}
  \frametitle{\texttt{std::vector}}
  \begin{itemize}
    \item Equivalent of Java's \texttt{ArrayList}
    \item \texttt{\#include <vector>}
    \item Is a template class (similar to Java generics)
    \item \link{http://www.cplusplus.com/reference/vector/vector/}{Official documentation}
  \end{itemize}
\end{frame}

\begin{frame}
  \frametitle{Java vs \cpp: Creation}
  \structure{Java}
  \code[language=java,font=\small,width=.95\linewidth]{arraylist-creation.java}
  \vskip2mm
  \structure{\cpp}
  \code[language=c++14,font=\small,width=.95\linewidth]{vector-creation.cpp}
\end{frame}

\begin{frame}
  \frametitle{Java vs \cpp: Indexing}
  \structure{Java}
  \code[language=java,font=\small,width=.95\linewidth]{arraylist-indexing.java}
  \vskip2mm
  \structure{\cpp}
  \code[language=c++14,font=\small,width=.95\linewidth]{vector-indexing.cpp}
\end{frame}

\begin{frame}
  \frametitle{Java vs \cpp: Setting}
  \structure{Java}
  \code[language=java,font=\small,width=.95\linewidth]{arraylist-setting.java}
  \vskip2mm
  \structure{\cpp}
  \code[language=c++14,font=\small,width=.95\linewidth]{vector-setting.cpp}
\end{frame}

\begin{frame}
  \frametitle{Java vs \cpp: Size}
  \structure{Java}
  \code[language=java,font=\small,width=.95\linewidth]{arraylist-size.java}
  \vskip2mm
  \structure{\cpp}
  \code[language=c++14,font=\small,width=.95\linewidth]{vector-size.cpp}
\end{frame}

\begin{frame}
  \frametitle{Java vs \cpp: Iterating}
  \structure{Java}
  \code[language=java,font=\small,width=.95\linewidth]{arraylist-iterating.java}
  \vskip2mm
  \structure{\cpp}
  \code[language=c++14,font=\small,width=.95\linewidth]{vector-iterating.cpp}
\end{frame}

\begin{frame}
  \frametitle{More Functionality}
  \begin{center}
    \begin{tabular}{ll}
      \textbf{Member function} & \textbf{Effect} \\
      \toprule
      \texttt{v.front()}       & First element \\
      \texttt{v.back()}        & Last element \\
      \texttt{v.data()}        & Returns underlying array (fragile!) \\
      \texttt{v.push\_back(x)} & Add to end \\
      \texttt{v.pop\_back()}   & Remove last \\
      \texttt{v.clear()}       & Remove all \\
    \end{tabular}
  \end{center}
\end{frame}


\section{Other Containers}

\frame{\tableofcontents[currentsection]}

\begin{frame}
  \frametitle{List}
  \begin{itemize}
    \item Doubly linked list
    \item \texttt{\#include <list>}
    \item \link{http://www.cplusplus.com/reference/list/list/}{Official documentation}
  \end{itemize}
  \begin{center}
    \begin{tabular}{ll}
      \textbf{Member function} & \textbf{Effect} \\
      \toprule
      \texttt{lst.front()}        & First element \\
      \texttt{lst.back()}         & Last element \\
      \texttt{lst.push\_front(x)} & Add to beginning \\
      \texttt{lst.pop\_front()}   & Remove first \\
      \texttt{lst.pop\_back()}    & Remove last \\
      \texttt{lst.push\_back(x)}  & Add to end \\
      \texttt{lst.clear()}        & Remove all \\
    \end{tabular}
  \end{center}
\end{frame}

\begin{frame}
  \frametitle{Sets}
  \structure{Set}
  \begin{itemize}
    \item Uses binary search tree (i.e.~no hashing)
    \item \texttt{\#include <set>}
    \item Cannot contain same element more than once
    \item \link{http://www.cplusplus.com/reference/set/set/}{Official documentation}
  \end{itemize}
  \vskip5mm
  \structure{Multiset}
  \begin{itemize}
    \item Uses binary search tree
    \item \texttt{\#include <set>}
    \item Can contain some element more than once
    \item \link{http://www.cplusplus.com/reference/set/multiset/}{Official documentation}
  \end{itemize}
\end{frame}

\begin{frame}
  \frametitle{Maps}
  \structure{Map}
  \begin{itemize}
    \item Uses binary search tree
    \item \texttt{\#include <map>}
    \item Associates one value with one key
    \item \link{http://www.cplusplus.com/reference/map/map/}{Official documentation}
  \end{itemize}
  \vskip5mm
  \structure{Multimap}
  \begin{itemize}
    \item Uses binary search tree
    \item \texttt{\#include <multimap>}
    \item Associates multiple values with one key
    \item \link{http://www.cplusplus.com/reference/map/multimap/}{Official documentation}
  \end{itemize}
\end{frame}

\begin{frame}
  \frametitle{Strings}
  \begin{itemize}
    \item Type of strings: \texttt{std::string}
    \item Strings also acts as containers
    \item \texttt{\#include <string>}
    \item \link{http://www.cplusplus.com/reference/string/basic_string/}{Official documentation}
  \end{itemize}
\end{frame}


\section{Iterators}

\frame{\tableofcontents[currentsection]}

\begin{frame}
  \frametitle{Iterators}
  \begin{itemize}
    \item \link{https://en.wikipedia.org/wiki/Iterator_pattern}{Iterator design pattern}
    \item Is an object that points to element of container
    \item It can be used to read and write to that location
          \begin{itemize}
            \item \texttt{x = *it} reads element at that position
            \item \texttt{*it = x} overwrites element at that position
          \end{itemize}
    \item It can be moved around
          \begin{itemize}
            \item \texttt{it++} moves to next element in container
            \item \texttt{it--} moves to previous element in container
          \end{itemize}
  \end{itemize}
\end{frame}

\begin{frame}
  \frametitle{Why Iterators?}
  \begin{itemize}
    \item Iterators abstract away the container specifics
    \item You can use vector, list, set, \dots iterators in the same way
    \item Iterators smart enough to know how to deal with the underlying container
  \end{itemize}
\end{frame}

\begin{frame}
  \frametitle{How To Get An Iterator}
  \begin{itemize}
    \item Container classes offer \texttt{begin()} and \texttt{end()}
    \item \texttt{begin()} gives iterator pointing to first element
    \item \texttt{end()} gives iterator pointing \emph{one past} last element
  \end{itemize}
\end{frame}

\begin{frame}
  \frametitle{Example: Zeroing Elements}
  \begin{itemize}
    \item You want to write \texttt{zero({\itshape container}<int>*)}
    \item Overwrites each integer in container with \texttt{0}
    \item You don't write it once for each container
          \begin{itemize}
            \item zero(\texttt{std::vector<int>*})
            \item zero(\texttt{std::list<int>*})
            \item zero(\texttt{std::set<int>*})
            \item \dots
          \end{itemize}
  \end{itemize}
\end{frame}

\begin{frame}
  \frametitle{Example: Zeroing Elements}
  \code[language=c++14,font=\small]{zero.cpp}
  \begin{itemize}
    \item \texttt{template} is \cpp's version of generics, but more flexible
    \item Basically says ``Give me \texttt{start} and \texttt{end} of \emph{any} type that
          \begin{itemize}
            \item can be compared using \texttt{!=}
            \item can be dereferenced using \texttt{*}
            \item can be increment with \texttt{++}
          \end{itemize}
  \end{itemize}
\end{frame}

\begin{frame}
  \frametitle{Example: Zeroing Elements}
  \code[language=c++14,font=\small]{zero-usage.cpp}
\end{frame}

\begin{frame}
  \frametitle{Why Not Inheritance?}
  \begin{itemize}
    \item Same could be achieved using subtyping
          \begin{itemize}
            \item \texttt{Iterator} superclass
            \item Each container defines own subclass
          \end{itemize}
    \item Using templates is more efficient
          \begin{itemize}
            \item More information at compile time
            \item Compiler can optimise better
          \end{itemize}
  \end{itemize}
\end{frame}

\begin{frame}
  \frametitle{Algoritms}
  \begin{itemize}
    \item Standard library provides functions on iterators
    \item \texttt{\#include <algorithm>}
    \item \link{http://www.cplusplus.com/reference/algorithm/}{Official documentation}
  \end{itemize}
  \begin{center}
    \begin{tabular}{ll}
      \textbf{Function} & \textbf{Effect} \\
      \toprule
      \texttt{all\_of}          & Checks if all elements satisfy condition \\
      \texttt{any\_of}          & Checks if any element satisfies condition \\
      \texttt{none\_of}         & Checks if no element satisfies condition \\
      \texttt{find\_if}         & Finds first element that satisfies condition \\
      \texttt{find\_if\_not}    & Finds first element that does not satisfy condition \\
      \texttt{count\_if}        & Counts elements that satisfy condition \\
      \texttt{count\_if\_not}   & Counts elements that do not satisfy condition \\
    \end{tabular}
  \end{center}
\end{frame}

\begin{frame}
  \frametitle{Example}
  \code[language=c++14]{all.cpp}
\end{frame}

\begin{frame}
  \frametitle{Example}
  \code[language=c++14]{any.cpp}
\end{frame}

\begin{frame}
  \frametitle{Example}
  \code[language=c++14]{count_if.cpp}
\end{frame}

\begin{frame}
  \frametitle{Reverse Iterators}
  \begin{itemize}
    \item Useful for iterating backwards
    \item Use \texttt{xs.rbegin()} and \texttt{xs.rend()} to get them
    \item Use \texttt{it.base()} to turn it back into a forward iterator
  \end{itemize}
  \code[language=c++14,width=.9\linewidth]{reverse-iterators.cpp}
\end{frame}

\end{document}



%%% Local Variables:
%%% mode: latex
%%% TeX-master: "containers"
%%% End:
